\chapter{Introduction}
\textbf{Cartoonizer} aims primarily to create a graphic effect in "cartoon" style applied to digital images. This effect is achieved by reducing the number of colors in the image using the \textbf{k-means clustering} method. This algorithm groups the colors of the image into a predefined number of clusters, allowing each pixel to be represented by the color of the centroid of the cluster it belongs to. The result is a simplified image, with sharper color transitions and a visually appealing appearance similar to an animated drawing.
The implementation of the Cartoonizer was developed using \textbf{CUDA} (\textit{Compute Unified Device Architecture}) technology, designed to leverage the parallel computing power of \textbf{GPUs} (\textit{Graphics Processing Units}), with the primary goal of significantly improving performance compared to a sequential version of the algorithm, enabling the processing of large images in reduced time.
A comprehensive evaluation will be presented in the report, showcasing the performance of the algorithm across various \textbf{NVIDIA} devices. The analysis will include comparisons based on different parameters, such as thread numbers and device configurations, providing detailed insights into the scalability and optimization of the algorithm. This section will delve into how varying these factors influences both execution time and overall performance. The results will be meticulously detailed, illustrating the impact of parallelization and hardware-specific optimizations in real-world scenarios.
The idea for this project stems from two foundational research papers that inspired its development. The first paper discusses various methods to generate cartoonized painterly effects on grayscale and colored images. It introduces the concept of vector quantization to achieve the painterly effect and compares algorithms such as LBG, KPE, and KMCG based on their processing time and visual results. These methods have applications in fields such as movie-to-comic conversions and digital art software \cite{image_cartoonization_methods}.
The second paper focuses specifically on the use of k-means for color quantization. While k-means has historically been viewed as computationally expensive and sensitive to initialization, the paper demonstrates that with efficient implementations and appropriate initialization strategies, k-means can serve as a highly effective color quantization method. The experiments conducted on diverse images highlight the algorithm’s performance and its potential in image processing \cite{improving_k_means}.
In the following chapters and sections, we will examine in detail the approach adopted, starting from the description of the problem and the techniques used, moving through the implementation in CUDA, and finally evaluating the performance achieved. The results highlight the benefits of parallelization, demonstrating how GPU computing can be utilized for creative and high-impact visual applications.

